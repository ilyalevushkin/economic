\documentclass[a4paper,14pt]{extarticle}

\usepackage[T2A]{fontenc}			
\usepackage[utf8]{inputenc}			
\usepackage[english,russian]{babel}

\usepackage[
bookmarks=true, colorlinks=true, unicode=true,
urlcolor=black,linkcolor=black, anchorcolor=black,
citecolor=black, menucolor=black, filecolor=black,
]{hyperref}

\usepackage{color}
\usepackage{caption}
\DeclareCaptionFont{white}{\color{black}}
\DeclareCaptionFormat{listing}{\colorbox{white}{\parbox{\textwidth}{#1#2#3}}}
\captionsetup[lstlisting]{format=listing,labelfont=white,textfont=white}

\usepackage{amsmath,amsfonts,amssymb,amsthm,mathtools} 
\usepackage{wasysym}

\usepackage{graphicx}
%\usepackage[cache=false]{minted}
\usepackage{cmap}
\usepackage{indentfirst}

\usepackage{listings} 
\usepackage{fancyvrb}

\usepackage{geometry}
\geometry{left=2cm}
\geometry{right=1.5cm}
\geometry{top=1cm}
\geometry{bottom=2cm}

\setlength{\parindent}{5ex}
\setlength{\parskip}{0.5em}

\usepackage{color}
\usepackage[cache=false, newfloat]{minted}
\newenvironment{code}{\captionsetup{type=listing}}{}
\SetupFloatingEnvironment{listing}{name=Листинг}
 
 
 \begin{document}
 	
 	\def\figurename{Рисунок}
 	
 	\begin{minipage}{0.2\textwidth}
 		\includegraphics[scale=0.05]{img/bmstu.png}
 	\end{minipage}
 	\begin{minipage}{0.7\textwidth}
 		\small
 		\begin{center}
 			\textbf{Министерство науки и высшего образования Российской Федерации}
 			
 			\textbf{Федеральное государственное бюджетное образовательное учреждение высшего образования «Московский государственный технический университет имени Н.Э. Баумана}
 			
 			\textbf{(национальный исследовательский университет)»}
 			
 			\textbf{(МГТУ им. Н.Э. Баумана)}
 		\end{center}
 	\end{minipage}
 	
 	\vspace*{5mm}
 	
 	\vspace*{30mm}
 	
 	\LARGE
 	\begin{center}
 		\textbf{Рубежный контроль №1}
 		
 		%	\textbf{к курсовому проекту на тему:}
 		\textbf{Реферат на тему:}
 		
 		\textbf{<<Концепция длинных волн (K-waves) Н.Д. Кондратьева. >>}
 	\end{center}
 	
 	%\huge
 	%\begin{center}
 	%	\textbf{Лабораторная работа №9}
 	%\end{center}
 	%
 	%\begin{center}
 	%	\textbf{Тема:} <<Обработчики прерываний>>
 	%\end{center}
 	
 	\vspace*{15mm}
 	
 	\large
 	\begin{flushleft}
 		\textbf{Дисциплина:} Экономика \\
 		\textbf{Студент:} Левушкин И. К. \\
 		\textbf{Группа:} ИУ7-72Б \\
 		%        \textbf{Оценка (баллы):} \\
 		\textbf{Преподаватель:} Герцик Ю.Г.\\
 	\end{flushleft}
 	
 	\vspace*{50mm}
 	
 	\large
 	\begin{center}
 		Москва, 2020 г.
 	\end{center}
 	
 	\thispagestyle{empty}
 	
 	\newpage
 	
 	\tableofcontents
 	\newpage
 	\section*{Введение}
 	\addcontentsline{toc}{section}{Введение}
 	
 	Цикличность как экономическую закономерность отрицают многие ученые-экономисты. Однако жизнь торжествует, и цикличность приковывает внимание наиболее пытливых исследователей.
 	
 	Тяжелый экономический кризис, завершивший период «военного коммунизма», был также и первым примером колебательного, неравномерного развития советской экономики. Однако сам факт возможности возникновения кризиса в советской экономической системе привел к тому, что ученые стали изучать не только проблемы неравномерности развития экономики вообще и в частности экономики страны, но и возникающие при этом противоречия и специфический механизм их разрешения, роль рынка и возможности управления.
 	
 	Не будет преувеличением утверждение, что особое место в работе теории цикличности принадлежит Николаю Дмитриевичу Кондратьеву. Признанием его заслуг в этой области служит то, что многие зарубежные ученые называют длинные волны его именем. Выпускник Юридического факультета Петербургского Университета Николай Дмитриевич Кондратьев еще в двадцатых годах открыл широкую дискуссию по проблемам длинных волн. Подлинно мировую известность ему принес доклад «Большие циклы конъюктуры», сделанный им на заседании ученого совета Института экономики в 1928 году.
 	
 	\textbf{Целью работы} является изучение того, что из себя представляют длинные волны по Кондратьеву и механизм их возникновения.
 	
 	\section{Основные положения гипотезы Н. Д. Кондратьева о существовании больших циклов экономической конъюнктуры}
 	
 	В начале 2О-х годов Кондратьев развернул широкую дискуссию по вопросу о длительных колебаниях при капитализме. В те времена еще очень сильны были надежды на скорую революцию в передовых капиталистических странах, и поэтому вопрос о будущем капитализма, о возможности нового его подъема, достижения им более высокой стадии развития был чрезвычайно актуален.
 	
 	Дискуссия началась с опубликованной в 1922 году работы "Мировое хозяйство и его конъюнктуры во время и после войны", в которой Кондратьев выступил с предположением о существовании длинных волн в развитии капитализма. Несмотря на отрицательную реакцию большинства советских ученых на эту публикацию, Н. Д. Кондратьев продолжал последовательно отстаивать свою позицию в следующих работах:
 	\begin{itemize}
 		\item "Спорные вопросы мирового хозяйства и кризиса (ответ нашим
 		критикам)" – 1923
 		\item "Большие циклы конъюнктуры" – 1925
 		\item "К вопросу о больших циклах конъюнктуры" – 1926
 		\item "Большие циклы конъюнктуры: Доклады и их обсуждение в Институте экономики" (совместно с Опариным Д.И.) – 1928
 	\end{itemize}
 	
 	Исследования и выводы Кондратьева основывались на эмпирическом анализе большого числа экономических показателей различных стран на довольно длительных промежутках времени, охватывавших 100-150 лет.
 	
 	Н. Д. Кондратьевым была проанализирована динамика изменения следующих показателей с конца XVIII века по начало XX века:
 	
 	\begin{enumerate}
 		\item по Англии: цен, процентов на капитал, заработной платы сельскохозяйственных и текстильных рабочих, внешней торговли, производства угля, чугуна, свинца.
 		\item по Франции: цен, процента на капитал, внешней торговли, потребления угля, посевной площади овса, портфеля Французского банка, вкладов в сберегательных кассах, потребления хлопка, кофе, сахара.
 		\item по Германии: производства угля и стали.
 		\item по США: цен, производства угля, чугуна и стали, количества веретен хлопчатобумажной промышленности, посевных площадей хлопка.
 		\item мирового производства угля и чугуна.
 	\end{enumerate}
 	
 	Показатели производства и потребления - не общие, а в расчете на душу населения.
 	
 	С помощью метода наименьших квадратов из ряда выделялись (в основном квадратичные) тренды, а затем полученные остатки осреднялись с помощью девятилетней скользящей средней. Осреднение позволяло сгладить колебания, происходящие чаще, чем раз в девять лет. Длина цикла оценивалась как расстояние между соседними пиками или спадами.
 	
 	Конечно, такая математическая методика исследования, применявшаяся Кондратьевым, не была лишена недостатков и подвергалась справедливой критике со стороны его оппонентов, но все возражения касались лишь точной периодизации циклов, а не их существования. Н. Д. Кондратьев понимал необходимость вероятностного подхода при исследовании статистических рядов экономических показателей. В своей статье "Большие циклы конъюнктуры" он писал, что считать доказанным наличие таких циклов нельзя, но вероятность их существования велика. Ни один из имеющихся методов математической статистики не может с достаточной степенью вероятности подтвердить присутствие 5О-ти летних циклов на отрезке 1ОО - 15О лет, т.е. на основании информации, содержащей максимум 2-3 колебания.
 	
 	Однако, возражая на заявления критиков, что нельзя говорить о
 	"правильности", то есть о периодичности больших циклов, поскольку их длительность колеблется от 45 до 60 лет, Кондратьев справедливо возражал, что большие циклы с вероятностной точки зрения не менее "правильны", чем традиционные циклические кризисы. Так как длина традиционного циклического кризиса варьируется в пределах от 7 до 11 лет, то его отклонение от средней составляет более 40\%, а такое отклонение от средней для большой волны, длительность которой изменяется от 45 до 6О лет, меньше 3O\%.
 	
 	Так как никакой математический аппарат анализа временных рядов не может с достаточной вероятностью подтвердить или опровергнуть существование длинных циклов, Кондратьев искал дополнительную информацию, стараясь найти свойства и явления, общие для соответствующих фаз обнаруженных им длинных циклов. К началу 2О-х годов мировой капитализм пережил, по расчетам Кондратьева две с половиной длинных волны, длиной 48 – 55 лет. На подавляющем большинстве кривых эти циклы четко видны безо всякой математической обработки. Периоды колебаний и основные (верхние и нижние) точки кривых зависимостей разных показателей совпадают (±3 года).
 	
 	
 	На протяжении всего исследуемого периода Кондратьевым также было сделано 4 важных наблюдения относительно характера этих циклов - «4 эмпирические правильности». Две из них относятся к повышательным фазам, одна к стадии спада и еще одна закономерность проявляется на каждой из фаз цикла.
 	
 	\begin{enumerate}
 		\item У истоков повышательной фазы или в самом ее начале происходит глубокое изменение всей жизни капиталистического общества. Этим изменениям предшествуют значительные научно-технические изобретения и нововведения. В повышательной фазе первой волны, то есть в конце XVIII века, это были: развитие текстильной промышленности и производство чугуна, изменившие экономические и социальные условия общества. Рост во второй волне, то есть в середине XIX века Кондратьев связывает со строительством железных дорог, которое позволило освоить новые территории и преобразовать сельское хозяйство. Повышательная стадия третьей волны в конце XIX и начале XX века, по его мнению, была вызвана широким внедрением электричества, радио и телефона. Перспективы нового подъема Кондратьев видел в автомобильной промышленности.
 		\item На периоды повышательной волны каждого большого цикла приходится наибольшее число социальных потрясений (войн и революций).
 		
 		Приведем список самых основных событий.
 		
 		\begin{itemize}
 			\item I повышательная волна : Великая французская революция, наполеоновские войны, войны России с Турцией, война за независимость США.
 			\item I понижательная волна : французская революция 1830 г., движение чартистов в Англии.
 			\item II повышательная волна : революции 1848-1849 гг. в Европе (Франция, Венгрия, Германия), Крымская война 1856 г., восстание сипаев в Индии 1867-1869 гг., гражданская война в США 1861-1865 гг., войны за объединение Германии 1865-1871 гг., французская революция 1871 г..
 			\item II понижательная волна : война России с Турцией 1877-1878 гг..
 			\item III повышательная волна : англо-бурская война 1899-1902 гг., русско-японская война 1904 г., первая мировая война, революции 1905 г. и 1917 г. и гражданская война в России.
 		\end{itemize}
 		
 		Ясно видно, что социальные потрясения повышательных волн намного превосходят таковые понижательных волн как по числу событий, так и (что более важно) по числу жертв и разрушений.
 		
 		\item Понижательные фазы оказывают особенно угнетающее влияние на сельское хозяйство. Низкие цены на товары в период спада способствуют росту относительной стоимости золота, что побуждает увеличивать его добычу. Накопление золота содействует выходу экономики из затяжного кризиса.
 		
 		\item Периодические кризисы (7-11-летнего цикла) как бы нанизываются на соответствующие фазы длинной волны и изменяют свою динамику в зависимости от нее - в периоды длительного подъема больше времени приходится на <<процветание>>, а в периоды длительного спада учащаются кризисные годы.
 	\end{enumerate}
 	
 	\section{Причины существования <<длинных волн>>}
 	
 	Хотя Кондратьевым был рассмотрен период длиной 140 лет (всего 2.5 длины волны большого цикла), он делает вывод (как уже было отмечено выше), что наличие таких циклов весьма вероятно, и их существование не может быть объяснено случайными величинами. По его мнению, их причины необходимо искать в особенностях, присущих капиталистической системе хозяйства. Однако при построении объяснения причин наличия таких циклов он встретился с очень большими трудностями. Его гипотезу о причинах этих циклов есть смысл привести целиком\cite{first}.
 	
 	<<Длительность функционирования различных созданных хозяйственных благ и производительных сил различна. Равным образом для их создания требуется различное время и различные средства. Как правило, наиболее длительный период функционирования имеют основные виды производительных сил. Они же требуют наибольшего времени и наибольших аккумулированных средств для их создания.
 	
 	Отсюда возникает необходимость для экономики понятия о различных видах равновесия применительно к различным периодам времени.
 	
 	Большие циклы можно рассматривать как нарушение и восстановление экономического равновесия длительного периода. Основная их причина лежит в механизме накопления, аккумулирования и рассеяния капитала, достаточного для создания новых основных производительных сил. Однако действие этой основной причины усиливает действие вторичных факторов.
 	
 	В соответствии с изложенным развитие большого цикла получает следующее освещение.
 	
 	Начало подъема совпадает с моментом, когда накопление и аккумулирование капитала достигает такого напряжения, при котором становится возможным рентабельное инвестирование капитала в целях создания основных производительных сил и радикального переоборудования техники.
 	
 	Начавшееся повышение темпа хозяйственной жизни, осложненное промышленно-капиталистическими циклами средней длительности, вызывает обострение социальной борьбы, борьбы за рынок и внешние конфликты.
 	
 	В этот период темп накопления капитала ослабевает и усиливается процесс рассеяния свободного капитала. Усиление действия этих факторов вызывает перелом темпа экономического развития и его замедление. Так как действие указанных факторов сильнее в промышленности, то перелом обычно совпадает с началом длительной сельскохозяйственной депрессии.
 	
 	Понижение темпа сельскохозяйственной жизни обусловливает, с одной стороны, усиление поисков в области усовершенствования техники, с другой - восстанавливает процесс аккумулирования в руках финансово-промышленных и других групп в значительной мере за счет сельского хозяйства.
 	
 	Все это создает предпосылки для нового подъема большого цикла, и он повторяется снова, хотя и на новой ступени развития производительных сил>>.
 	
 	\section{Эндогенный механизм длинных волн по Н. Д. Кондратьеву}
 	
 	
 	Н. Д. Кондратьев в своей работе <<Длинные волны конъюнктуры>> пишет, что волнообразные движения представляют собой процесс отклонения от состояний равновесия, к которым стремится капиталистическая экономика. Он ставит вопрос о существовании нескольких равновесных состояний, а отсюда и о возможности нескольких колебательных движений. Кондратьев предлагает говорить не только о кризисах, но исследовать всю совокупность волнообразных движений при капитализме, то есть разрабатывать общую теорию колебаний.
 	
 	Согласно Кондратьеву существует три вида равновесных состояний:
 	
 	\begin{enumerate}
 		\item Равновесие <<первого порядка>> - между обычным рыночным спросом и предложением. Отклонения от него рождают краткосрочные колебания периодом 3 - 3,5 года, то есть циклы в товарных запасах.
 		\item Равновесие <<второго порядка>>, достигаемое в процессе формирования цен производства путем межотраслевого перелива капитала, вкладываемого главным образом в оборудование. Отклонения от этого равновесия и его восстановление Кондратьев связывает с циклами средней продолжительности.
 		\item Равновесие <<третьего порядка>> касается <<основных материальных благ>>. В эту категорию Кондратьев включает промышленные здания, инфраструктурные сооружения, а также квалифицированную рабочую силу, обслуживающую данный технический способ производства. Запас «основных капитальных благ» должен находиться в равновесии со всеми факторами, определяющими существующий технический способ производства, со сложившейся отраслевой структурой производства, существующей сырьевой базой и источниками энергии, ценами, занятостью и общественными институтами, состоянием кредитно-денежной системы и т.д.
 	\end{enumerate}
 	
 	Периодически это равновесие также нарушается и возникает необходимость создания нового запаса <<основных капитальных благ>>, которые бы удовлетворяли складывающемуся новому техническому способу производства. По Кондратьеву такое обновление <<основных капитальных благ>>, отражающее движение научно-технического прогресса, происходит не плавно, а толчками и является материальной основой больших циклов конъюнктуры.
 	
 	Здесь следует пояснить. В зарубежной литературе сложилось мнение, что в части, касающейся форм развития научно-технического прогресса, концепция Кондратьева близко подходит к инновационной теории длинных волн, разработанной Дж. Шумпетером. Основной претензией некоторых западных авторов к Кондратьеву является то, что он не отдал должное иновационной теории длинных волн и не развил ее. Они считают, что только из-за страха перед критикой со стороны советской экономической науки того времени и необходимости приспособления к политической обстановке он отклонился от углубленной разработки инновационного подхода и вместо этого ввел в свое объяснение длинного цикла марксистские концепции возмещения основного капитала и роли ссудного процента.
 	
 	Прежде всего, представляется некорректным критиковать ученого за то, что он не смог или не успел сделать, причем явно не по своей вине. Вспомним, что Кондратьев был в неполных 40 лет насильственно оторван от научной деятельности.
 	
 	Но главное не в этом. Думается, Кондратьев не пошел по пути Шумпетера, прежде всего вследствие собственных научных убеждений. В отличие от Шумпетера он искал объяснение длинным волнам не в готовности предпринимателей к инновациям и не в преходящих всплесках предпринимательской активности, а, прежде всего в самих основах воспроизводственного процесса.
 	
 	Согласно марксовой теории материальной основой периодичности кризисов является средний срок жизни оборудования. Кондратьев искал основу правильности длительных колебаний в схожих материальных факторах. Так поступил и марксист Де Вольф, который обратился к основному капиталу в транспортной инфраструктуре. Однако, в отличие от Де Вольфа, Кондратьев расширил материальную основу длинных волн, включив в нее – через необходимость сохранения равновесия третьего порядка – всю сумму капитала и трудовых ресурсов, обеспечивающих на длительной основе данный технический способ производства. Таким образом, он непосредственно подошел к понятию жизненного цикла технического способа производства, хотя и не употреблял этого более современного для нас термина.
 	
 	Но продолжим кондратьевское объяснение механизма длинных волн. Обновление и расширение <<основных капитальных благ>>, происходящее во время повышательной фазы длинного цикла радикально изменяют и перераспределяют производительные силы
 	общества. Для этого требуются огромные ресурсы в натуральной и денежной форме. Они могут существовать только в том случае, если были накоплены в предшествующей фазе, когда сберегалось больше, чем инвестировалось.
 	
 	В фазе подъема постоянный рост цен и заработной платы порождал у населения тенденцию больше расходовать, в период спада, наоборот падают цены и заработная плата. Первое ведет к стремлению сберегать, а второе - к снижению покупательной способности. Аккумуляция средств происходит также за счет падения инвестиций в период общего спада, когда прибыли становятся низкими и возрастает риск банкротства.
 	
 	Можно заметить, что такие явления имели место в капиталистической экономике в 8О-х годах, когда наблюдался отлив капиталов из производственной сферы в сферу спекулятивных биржевых операций. Даже в нашей стране, несмотря на то, что говорить о капиталистической системе преждевременно и учитывая специфику политической ситуации и налоговой системы, можно, тем не менее, проследить подобную ситуацию.
 	
 	Снижение товарных цен по Кондратьеву приводит к росту относительной стоимости золота. Возникает стремление увеличить его добычу. Появление дополнительного денежного металла способствует росту свободного ссудного капитала, и, когда его накапливается достаточное количество, рождается возможность новой радикальной перестройки хозяйства.
 	
 	Таким образом, основные элементы внутреннего эндогенного механизма длинного цикла по Кондратьеву таковы:
 	
 	\begin{enumerate}
 		\item Капиталистическая экономика представляет собой движение вокруг нескольких уровней равновесия. Равновесие <<основных капитальных благ>> (производственная инфраструктура плюс квалифицированная рабочая сила) со всеми факторами хозяйственной и общественной жизни определяет данный технический способ производства. Когда это равновесие нарушается, возникает необходимость в создании нового запаса капитальных благ.
 		\item Обновление <<основных капитальных благ>> происходит не плавно, а толчками. Научно-технические изобретения и нововведения при этом играют решающую роль.
 		\item Продолжительность длинного цикла определяется средним сроком жизни производственных инфраструктурных сооружений, которые являются одним из основных элементов капитальных благ
 		общества.
 		\item Все социальные процессы - войны, революции, миграции населения - результат преобразования экономического механизма.
 		\item Замена <<основных капитальных благ>> и выход из длительного спада требуют накопления ресурсов в натуральной и денежной форме. Когда это накопление достигает достаточной величины, возникает возможность радикальных инвестирований, которые выводят экономику на новый подъем.
 	\end{enumerate}

 	\section{Заслуга Кондратьева и современное значение его теории <<длинных волн>> в экономике}
 	
 	Многие видные экономисты в течение полувека разрабатывали и создавали свои концепции <<длинных циклов Кондратьева>>. Почему же именно Кондратьева? Ответ на этот вопрос можно обосновать следующими соображениями:
 	\begin{enumerate}
 		\item Появившиеся в печати материалы по длительным колебаниям до Кондратьева были единичными и носили лишь характер догадок. Кондратьев же рассматривал этот вопрос более обстоятельно. Кроме того, его работы переводились на английский, немецкий, французский языки.
 		\item Наибольшей научной заслугой Кондратьева является то, что он осуществил попытку построить замкнутую социально-экономическую систему, генерирующую внутри себя эти длительные колебания. В работах же предшественников Кондратьева обязательно присутствуют факторы, играющие роль внешнего толчка в формировании колебаний. Кондратьев же раскрывает внутренний механизм как спадов, так и подъемов. Именно это второе обстоятельство привлекло западных экономистов в то время, когда общая экономическая ситуация, особенно в 30-х годах, казалась безысходной. То есть кондратьевская концепция давала надежду на выход из большого кризиса.
 		\item Несомненным вкладом Н. Д. Кондратьева в современную эконометрическую науку было введение им вероятностных законов в анализ экономических процессов.
 		\item Кондратьев одним из первых поставил вопрос о существовании равновесий в экономике.
 		\item Привлекательным было также сочетание у Кондратьева экономического анализа с социологическим: до Кондратьева исследователи длительных колебаний больше внимания уделяли изучению материальных факторов, а Кондратьев рассматривал социальные и политические аспекты – войны, перевороты. Интересно, что он впервые ввел различие между <<промежуточными войнами>>, играющими роль стимулятора экономики в начале фазы подъема, и <<окончательными войнами>> и переворотами в конце подъема, разрешающими противоречие, накопившиеся в период подъема.
 		\item За рубежом имя Н. Д. Кондратьева никогда не забывали, и <<кондратьевские волны>> стали толчком к рождению целого направления в современной экономической науке. Оно бурно развивается и сегодня, поскольку резко ускорившийся научно-технический прогресс стал, похоже, сжимать <<длинные волны>>, и человечеству видимо, надо готовиться к серьезным колебаниям экономического развития.
 	\end{enumerate}
 	
 	Теоретические концепции длинных волн важны тем, что они дают необходимую основу для оценки состояния экономики и прогнозирования ее будущего состояния.
 	
 	\newpage
 	\section{Заключение}
 	\addcontentsline{toc}{section}{Заключение}
 	
 	В рамках данного реферата были изучено такое понятие как длинные волны по Кондратьеву. Были рассмотрены основные положения его гипотезы, а также причины их существования.
 	
 	Проанализировав изученную информацию можно сделать вывод, что концепция длинных волн Кондратьева является общепризнанным вкладом в развитие экономической науки. Эти концепции входят в основу построения различных оценок состояния экономики и прогнозирования ее будушего состояния.
 	
 	Таким образом, можно считать поставленную цель выполненной.
 	
 	\newpage
 
 	
 	\addcontentsline{toc}{section}{Список используемой литературы}
 	\begin{thebibliography}{4}
 		\bibitem{first}
 		Кондратьев Н. Д. <<Проблемы экономической динамики>>, М., 1989. [Электронный ресурс]. – Режим доступа: 
 		https://www.socionauki.ru/book/files/k\_waves/volume\_4/054-064.pdf, 
 		свободный – (21.09.2020)
 		
 		\bibitem{second}
 		Меньшиков С. М., Клименко Л. А. <<Длинные волны в экономике. Когда общество меняет кожу>>, М., 1989. [Электронный ресурс]. – Режим доступа: 
 		http://www.pseudology.org/Bank/DlinnyeVolny.pdf, 
 		свободный – (21.09.2020)
 		
 		\bibitem{third}
 		Игорь Липсиц <<Экономика без тайн>>, М., 1993. [Электронный ресурс]. – Режим доступа: https://klex.ru/me8, свободный - (21.09.2020)
 		
 		\bibitem{fourth}
 		Из истории экономической мысли //Экономика, 7/1990. [Электронный ресурс]. – Режим доступа: http://ecsocman.hse.ru/data/2018/08/04/1251869274/JIS\_10.1\_1.pdf, 
 		свободный – (21.09.2020)
 		
 	\end{thebibliography}
 
\end{document}